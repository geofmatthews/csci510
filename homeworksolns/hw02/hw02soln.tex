\documentclass{article}
\usepackage[margin=1in]{geometry}
\usepackage{amsmath}
\title{CSCI 510, Fall 2016, Homework \# 2}
\author{SOLUTION SKETCHES}
\date{Due date: Friday, October 21, Midnight} 
%\newcommand{\duple}[2]{\left[\begin{array}{c} #1\\#2\end{array}\right]}
\newcommand{\duple}[2]{\left[{#1}\atop{#2}\right]}
\begin{document}
\maketitle
\begin{enumerate}
\item Let $A/B=\{w| \mbox{$wx\in A$ for some $x\in B$}\}$.  Show that
  if $A$ is context free and $B$ is regular then $A/B$ is context-free.

  {\bf Solution:}
  If $A$ is context free and $B$ is regular we can assume there is a
  PDF for $A$ and a $DFA$ for $B$.  We will build a $PDA$ for $A/B$ by
  first running the $A$ machine over the string, and, at the end,
  running  a machine that does not consume input, but
  nondeterministically runs through the $B$ machine until it gets to
  an accepting state.

  The devil is in the details.
  Let
  \[
  P_A = \{Q_A, \Sigma_A, \Gamma_A, \delta_A, q_{A0}, F_A\}
  \]
  be a PDA for $A$, and let
  \[
  D_B = \{Q_B, \Sigma_B, \delta_B, q_{B0}, F_B\}
  \]
  be a DFA for $B$.  We now build a PDA for $A/B$,
  \[
  P_{A/B} = \{Q_{A/B}, \Sigma_{A/B}, \Gamma_{A/B}, \delta_{A/B}, q_{A/B0}, F_{A/B}\}
  \]
  as follows:
  \begin{align*}
    Q_{A/B} &= Q_A \cup  ( Q_A\times Q_B )\\
    \Sigma_{A/B} &=     \Sigma_{A} \cup    \Sigma_{B} \\
    \Gamma_{A/B} &= \Gamma_A\\
    q_{A/B0} &= q_{A0} \\
    F_{A/B} &= \{(f_a,f_b) | f_a\in F_A \mbox{ and } f_b \in F_B\}\\
  \end{align*}
  $\delta_{A/B}$ will consist of all transitions  $\delta_A$, together
  with all rules of the following types two types:
  \begin{enumerate}
    \item For each $q_A \in Q_A$ add
      the rule
      \[
      \delta_{A/B}(q_A,\varepsilon,\varepsilon) = ((q_A,q_{B0}),
      \varepsilon)
      \]
      This will allow us to jump, at the end of the input, from the
      $Q_A$ part of $P_{A/B}$ to the $Q_A\times Q_B$ part, and get
      started looking for the (invisible) ``$B$'' part of the string.
      The input and stack are untouched.

    \item
      Whenever
      \[
      \delta_A(q_{A1}, b, c) = (Q_{A2}, c)
      \]
      and
      \[
      \delta_B(q_{B1}, d) = q_{B2}
      \]
      then add the rule
      \[
      \delta_{A/B}((q_{A1},q_{B1}), \epsilon, b) = ((q_{A2},q_{B2}),
      c)
      \]
      This will allow us to simulate both the operation of the $A$
      machine and the $B$ machine simultaneously, and have the same
      effect on the stack that the $A$ machine would have had if the
      ``$B$'' part of the string had been on the input.
  \end{enumerate}
  To accept, we must end up accepting the pseudo-extension to the
  string with the $B$ machine, and accept the string together with the
  pseudo-extension with the $A$ machine.
  
\item For any language $A$, let $\mbox{\sl suffix}(A) =
  \{v|\mbox{$uv\in A$ for some string $u$}\}$.  Show that the class of
  context-free languages is closed under the {\sl suffix} operation.

  {\bf Solution:}

  This solution is very similar to the previous one, except that we
  run a ``no-input'' PDA with $\varepsilon$ transitions (similar to
  the ones in the $A$ PDA) that sets up what the stack would have been
  had we run $u$ (a random string), then nondeterministically jump,
  using $\varepsilon$ transitions, to the full machine to finish the
  job.  Lots of details omitted here.


\item Show that if $G$ is a CFG in Chomsky normal form, then any
  string $w\in L(G)$ of length $n\geq 1$, exactly $2n-1$ steps are
  required for any derivation of $w$.  Give a proof by induction.

{\bf Solution}

  First we prove a handy lemma by induction:

  \begin{description}
  \item[Lemma]
    
    Given a derivation of $w$ length $\ell$:
      \[
      s_0 \Rightarrow  s_1 \Rightarrow s_2 \Rightarrow s_3 \Rightarrow \ldots
      \Rightarrow  s_i \Rightarrow \ldots \Rightarrow s_{\ell}
      \]
      where $s_1$ is the start symbol, and $s_{\ell}=w$.  For each $i\in
      0\ldots \ell$ let $j_i$ be the number of times an $A\rightarrow a$
      type rule is applied in the dervation up to $s_i$, and $k_i$ be the
      number of times a $A\rightarrow BC$ type rule is applied in the
      derivation up to $s_i$.  Then
      \begin{align}
        |s_i| &= k_i+1
      \end{align} 
      
      {\bf Proof of Lemma (induction)}
      \begin{description}
        \item[Base:] $i=0$.  Then the derivation is just $s_0$, which
          is the start symbol with length 1, and no rules have been
          applied, so $k_0=0$, and
          $|s_0| = 1  = k_0+1$.
        \item[Step:] Assume $|s_i| = k_i+1$ and consider the step from
          $s_i$ to $s_{i+1}$.  We have two cases:
          \begin{enumerate}
          \item The rule applied at step $i$ is of the form
            $A\rightarrow a$.  Then $|s_{i+1}| = |s_i|$ and
            $k_{i+1}=k_{i}$ and so $|s_{i+1}| = k_{i+1}+1$
          \item The rule applied at step $i$ is of the form
            $A\rightarrow BC$.  Then $|s_{i+1}|  = |s_{i}|+1$ and
            $k_{i+1}=k_{i}+1$ and so $|s_{i+1}| = k_{i+1}+1$
          \end{enumerate}
      \end{description}
    \end{description}
\hfill
  \rule{1em}{1em}      
    
  {\bf Proof}

Using the same notation as in the lemma, clearly, for $i=0\ldots \ell$:
  \begin{align}
    i &= j_i + k_i
  \end{align}
Further, the last element in the derivation is all terminals.  Each
terminal is produced by one application of a rule of the form
$A\rightarrow a$, and so $j_{\ell} = n = |w|$.  Putting this together
with our lemma gives
  \begin{align}
    \ell  &= j_{\ell} + k_{\ell}\\
    &= n + k_{\ell}\\
    &= n + |s_\ell| - 1\\
    &= n + n - 1\\
    &= 2n-1
  \end{align}
\hfill
  \rule{1em}{1em}      



\end{enumerate}
\end{document}

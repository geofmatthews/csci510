\documentclass{article}
\usepackage[margin=1in]{geometry}
\title{CSCI 510, Fall 2016, Homework \# 3}
\author{YOUR NAME HERE}
\date{Due date: Wednesday, October 26, Midnight} 

\begin{document}
\maketitle
\begin{enumerate}
\item Give an implementation-level description of a Turing machine
  that recognizes the language $A$ over the alphabet $\{0,1\}$, where
  \[
A=  \{w | \mbox{$w$ does not contain the same number of 0s and 1s}\}
\]

\item Show that a language is decidable iff some enumerator enumerates
  the language in the standard string order.

\item A {\bf\em queue automaton} is like a push-down automaton except
  that the stack is replaced by a queue.  A {\bf\em queue} is a tape
  allowing symbols to be written only on the left-hand end and read
  only at the right-hand end.  Each write operations (we'll call it a
  {\em push}) adds a symbol to the left-hand end of the queue and each
  read operation (we'll call it a {\em pull}) reads and removes a
  symbol at the right-hand end.  As with a {\sf PDA}, the input is
  placed on a separate read-only input tape, and the head on the input
  tape can move only from left to right.  The input tape contains a
  cell with a blank symbol following the input, so that the end of the
  input can be detected.  A queue automaton accepts its input by
  entering a special accept state at any time.  Show that a language
  can be recognized by a deterministic queue automaton iff the
  language is Turing-recognizable. 

\end{enumerate}
\end{document}

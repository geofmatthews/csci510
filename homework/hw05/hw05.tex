\documentclass{article}
\usepackage[margin=1in]{geometry}
\title{CSCI 510, Fall 2016, Homework \# 5}
\author{YOUR NAME HERE}
\date{Due date: Wednesday, November 30, Midnight} 

\begin{document}
\maketitle
\begin{enumerate}
\item Show that $A$ is Turing-recognizable iff $A\leq_m A_{TM}$.

\item Let $J=\{w\ |\ \mbox{either $w=0x$ for some $x\in A_{TM}$, or
  $w=1y$ for some $y\in \overline{A_{TM}}$}\}$.  Show that neither $J$
  nor $\overline{J}$ is Turing-recognizable.

\item Let
  \[
  f(x) = \left\{\begin{array}{ll}
  3x+1 & \mbox{ for odd $x$}\\
  x/2 & \mbox{ for even $x$}
  \end{array}
  \right.
  \]
for any natural number $x$.  If you start with a natural number $x$
and iterate  $f$, you obtain a sequence, $x, f(x),
f(f(x)),\ldots$ Stop if you ever hit 1.  For example, if $x=17$, you
ge the sequence $17,52,26,13,40,20,10,5,16,8,4,2,1$.  Extensive
computer tests have shown that every starting point between 1 and a
very large positive integer gives a sequence that ends in 1.  But the
question of whether all positive starting points end up at 1 is
unsolved;  it is called the $3x+1$ problem.

Suppose that $A_{TM}$ were decidable by a TM $H$.  Use $H$ to describe
a TM that is guaranteed to state the answer to the $3x+1$ problem.

\item Let $T=\{\langle M \rangle | \mbox{$M$ is a TM that accepts
  $w^R$ whenever it accepts $w$}\}$.  Show that $T$ is undecidable.

\item Show that the Post Correspondence Problem is decidable over the
  unary alphabet $\Sigma = \{1\}$.

\item Prove that there exists an undecidable subset of $\{1\}^*$.


\end{enumerate}
\end{document}
